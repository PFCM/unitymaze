\documentclass[11pt]{article}
\usepackage[a4paper,margin=2.7cm]{geometry}

\usepackage{titlesec}
\usepackage{color}
\usepackage{graphicx}
\usepackage{longtable}
\usepackage{tabu}
\usepackage{fontspec}
\setmonofont[Scale=0.95]{Source Code Pro Light}
\setsansfont[ItalicFont={Helvetica Neue Thin Italic}]
			{Helvetica Neue Thin}

\usepackage{microtype}

\usepackage{amsmath}

\titleformat{\section}
	{\large\itshape}{\thesection 
					\hspace{.2em}
					  \textnormal{|}}{1em}{}
\titleformat{\subsection}
	{\itshape}{\thesubsection
					\hspace{.2em}
					\textnormal{|}
					  }{1em}{}

\titleformat{\subsubsection}
			{\itshape}{\thesubsubsection
						\hspace{.2em}
						\textnormal{|}
						}{1em}{\normalfont}




\title{\sffamily{\Huge COMP313}\\ Assignment Log}
\author{\sffamily Paul Mathews \emph{300190240}}
\date{}

\begin{document}

\maketitle

\section{Log}

\begin{longtabu}{X[r] | X[3,l]}
\hline
First few weeks & Spent a while making mazes because mazes are fun \& not a bad way to 
				  start learning a new language. Made a prefab that is a square maze 
				  generated at runtime. Needed to figure out how to load a prefab by name,
				  this is easy if it is in a folder called ''Resources''. \\
				& Managed to get the prefab loading new instances of itself. Added the rolling ball from
				  the Unity tutorial to bounce around the maze and an invisible cube which has the script
				  which loads a new maze and positions it so the corners line up. As the entrance of the 
				  mazes is fixed, this just needs to position it with a constant offset which is easy. It
				  would be cooler (and probably necessary for the big project) to have it query the new 
				  maze for its door position and line them up based on that. Did a little bit of work on
				  this, making a base Room class which extends MonoBehaviour and has a method to query for
				  the entrance position. Individual rooms/mazes can have a script that extends this.\\
	\hline
Upon receipt of model & Started doing player controller -- fiddled with the built in Unity
					    one but wanted to add in jumping (because it would be cool to be able to 
					    jump over the walls) and also get to grips with Unity a bit more so wrote it
					    more-or-less from scratch. Implemented a jump by adding a big upwards force
					    to the RigidBody as I wanted it to look reasonable. This caused some issues if 
					    you land partially on the walls as the character tends to bounce around confusingly
					    and then get stuck upside down etc. A fairly hacky solution was to give the player
					    unit a very large mass (and dial up the jump force correspondingly) which leads to 
					    a bit less rolling round (and less getting bumped around by the AI, see later).\\
	\hline
Shortly afterwards    & Added in AI character. Read the requirements backwards and started off by trying to
						make it chase the player. As we are operating in a randomly generated maze it seemed
						like a good idea to offload the pathfinding to Unity (not that A* would be that hard
						especially given we have the maze data structure, but if Unity will do it for us
						then why not). This works pretty well just add a NavMesh to the ground plane and 
						NavMeshObstacles to all the walls. The issue happens when a new maze is loaded. 
						Unity only allows one NavMesh per scene and refuses to modify it at runtime. If 
						you need to load things additively then tough luck. The only way a NavMesh can be 
						modified on the fly is with NavMeshObstacles. Fortunately the walls of the maze are 
						NavMeshObstacles, so the solution is to make a gigantic invisible ground plane so 
						that there is an enormous area of flat, square NavMesh. Then when new mazes are 
						loaded on top of it the walls carve out holes in it and all is well. Unfortunately 
						this is not a good solution. Firstly it limits the size of the game from infinite
						to however big the NavMesh is. Secondly it is not clear how much space this giant
						NavMesh takes in memory -- hopefully not a lot because until the mazes are loaded it
						is only 4 vertices but you never know. Thirdly it only works because all the mazes 
						are flat. If you wanted to load anything with stairs or ramps this would fail; it 
						only works if the only modification to the NavMesh upon loading a new level/maze is
						carving out holes.\\
	\hline
10/4/15 & Had some problems with the AI character bumping into the player irritatingly. Ended up adding a 
		  second, large, spherical collider to the player to use as a trigger. This gives a good bit of 
		  space around the player to fire off the AI's OnTriggerEnter while it has enough time to stop.
		  Might also help when interacting with objects.\\
	\hline
13/4/15 & Added some objects to interact with implemented as a prefab with a tag. The tag is the important
		  part, any object with the tag ''pickup'' will be treated the same by the code. This seems like a 
		  flexible way to add new items or change them around. \\
		& Wanted to be able to bump the items, thought it would be cool to give them a kick by applying a
		  force at an angle. Needed to figure out if there were any objects in front of the play and if so 
		  get a hold of the nearest in code. Could use the above big collider, but would have had to 
		  balloon it out a bit further than I wanted to keep the AI nice. Physics.Raycast seemed like a
		  good plan: cast a ray out straight ahead and if it hits something we are good go. It even 
		  provides a way to limit the distance which is perfect. In practice this required a lot of 
		  tweaking. Firstly it won't return an object you are colliding with, so had to add some code to 
		  OnTriggerEnter to keep track of pickups being collided with and in the case the raycast fails, 
		  check to see if we've hit something that is still close enough to interact with. Made a few silly
		  errors just trying to get the logic right for this but got there in the end...\\
		& Now \texttt{SpiderController.GetObjectInFront ()} is almost perfect except that I haven't gotten
		  around to ensuring that the collided object is actually in front of the player's spider. This
		  shouldn't be too hard, just have get a vector between the two and check the angle between that 
		  and a vector straight out of the player (which can be found with 
		  \texttt{transform.forward}).\\ 
		& \emph{(did this on 19/4 using \texttt{Vector3.Angle}, gave it 15 degrees either way)}\\
	\hline
17/4/15 & Let the user actually pick up the pickups (with left shift). No real issues here, keeps a 
		  keeps a list of the actual GameObjects, deactivated. If you press the pickup button again it
		  throws the last object picked up. UI is fairly basic, just showing a count of how many things
		  have been picked up so far. Using the actual GameObjects means any of their state is maintained,
		  this can only be a good thing. Ideally would query them for an icon or something and display
		  that.\\
	\hline
18/4/15 & Did the AI state machine properly. Has 4 states: patrolling, looking, chasing and found. \\
		& Patrolling cycles through an array of positions. This is exposed as a public array of Transforms
		  so any number of positions can be set up in the editor. In this state it will go to each 
		  position in turn, starting again from the beginning once it reaches the last one. Navigation
		  and movement is handled by the NavMeshAgent, this part is pretty simple and easy. \\
		& While patrolling the agent is constantly polling for nearby pickups. When one is within a 
		  specified distance, the agent stops and looks at it until it gets too far away. Stopping and
		  starting the navigation was a little bit odd -- \texttt{NavMeshAgent.Stop ()} didn't seem to 
		  actually do anything and the agent would only stop if the agent was disabled as well. This meant
		  to start it up again it had to be re-enabled before calling any other methods or it would throw 
		  an error.\\
		& When patrolling (but not while distracted by an object) the agent also checks to see how far 
		  away the player is. If the player gets too close, the agent's state changes to chasing. When
		  chasing, the NavMeshAgent constantly has its destination set to the player's position. Getting
		  it to stop appropriately was tricky as setting the NavMeshAgent's stopping distance property 
		  did not seem to change very much. The solution was, again, to explicitly disable navigation
		  when the agent collides with the player.\\
		& When the player collides with the agent (recall that the player has slightly oversized collider
		  for precisely this purpose, so it actually slightly before the models touch) the agent starts to
		  chase, regardless of whether it was patrolling or looking at an object. If the agent was chasing
		  and collides, it moves to the ''found'' state, which simply disables navigation and looks at the 
		  player until they get far enough away to start chasing again. \\
		& While chasing it is possible to distract the agent with a pickup object. It still looks for 
		  these and if they get close enough it will go into the ''looking'' state. The only way out of 
		  this is back into patrolling. In this case the patrol should pick up where it left off, going to 
		  the waypoint it was going towards before it started chasing. It will not pick up the exact path,
		  although if you had sufficiently finely grained waypoints it would appear to.\\
	\hline
19/4/15 & Tidied up, commented out debugging print statements etc. Also fixed the pickups so that objects
		  being touched will be picked up only if they are within 25 degrees either side of wherever the
		  player is facing to make it easier to pick up a ball you are pushing in front of you. Also 
		  changed the inventory to display a 'O' for each ball picked up (rather than a count). This was
		  done by using a UIText object and just changing the text. It is in a box so that if you pick up 
		  more than 6 it will start a new line.\\
		& Was finding the camera awkward, being the child of the player GameObject meant that on the rare
		  occasions when the player jumps and gets flipped around ending up upside down or sideways 
		  the camera would follow suit. Moved it outside and wrote a short script, based off the one from 
		  the Rolling Ball tutorial, which follows the player around. Making sure it rotated correctly was
		  a little bit awkward, in the end the simplest way was to separate the offset on the x,z plane
		  and the y axis and constantly set the camera to be offset by the x,z distance behind the player
		  and then use \texttt{Transform.LookAt} to keep the player centered.\\
		& Added a state and a method to make the AI character just move to a specific place. Still uses
		  the NavMesh, so it's really only a few lines. Seemed silly to implement a whole search and some 
		  kind of data structure to keep track of how to go where when Unity will just do it for you
		  (as long as the NavMesh is alright, which may not be the case all the time if you want to load
		  content additively).\\
	\hline
\end{longtabu}

\section{Summary}
\subsection{Controls}
\begin{tabu}{X[r] | X[l,2.5]}
\textit{Key} & \textit{Action} \\
\hline
w/up & move forward \\
s/down & move backward \\
a/left & turn left \\
d/right & turn right\\
f/left click & bump an object in front \\
left shift/right click & pickup an object in front, throw an object if none in front and at least one 
held\\
space & jump -- be careful with this, if you land badly and get flipped around it can be very hard to 
right yourself.\\
\end{tabu}

\subsection{Brief Description}

The player controls a spider-like character that starts in a corner of a square maze. There is a
computer controlled spider that starts in a different corner and will initially be attempting to visit
all four corners in turn. There are sparkly spherical objects in several locations on the initial maze.
If the computer controlled character nears one it will stop and look at it until it gets too far away
at which point it will return to patrolling. The player can push them around, pick them up and throw them.
If the player gets close to the AI while it is not distracted or bumps it at any time then the AI will 
start to chase the player. The only way to stop it is to distract it by putting one of the sparkly objects
in its path.

If any object or spider makes it to the back-left corner of the maze, where there is a gap in the outer
wall, a new adjoining square of maze will be generated. This new maze contains no extra pickups, but new
mazes can be generated infinitely. The player can jump over the internal walls of the maze, but the AI 
will respect them. It is a good idea to be careful when jumping around as the spider has a tendency to 
flip if you land on a corner or a ball.


\end{document}